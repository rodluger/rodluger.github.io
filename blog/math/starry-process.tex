Let
$\mathbf{f} = \left( f_0 \, f_1 \, \cdots \,  f_K \right)^\top$
denote a vector of $K$ flux measurements at times
$\left( t_0 \,  t_1 \,  \cdots \, t_K \right)^\top$; i.e., an observed
light curve.
Conditioned on certain physical properties of the star, $\pmb{\theta}$,
we wish to compute the mean $\pmb{\mu}(\pmb{\theta})$ and
covariance $\pmb{\Sigma}(\pmb{\theta})$
of $\mathbf{f}$, which together fully specify our Gaussian process.
As with any random variable, the mean and covariance may be computed from
the expectation value of $\mathbf{f}$ and
$\mathbf{f}\,\mathbf{f}^\top$, respectively:
\begin{align}
    \label{eq:mean}
    \pmb{\mu}(\pmb{\theta})
     & = \mathrm{E} \Big[ \mathbf{f} \, \Big| \, \pmb{\theta} \Big]
    \\
    \label{eq:cov}
    \pmb{\Sigma}(\pmb{\theta})
     & = \mathrm{E} \Big[ \mathbf{f} \, \mathbf{f}^\top \, \Big| \, \pmb{\theta} \Big] - \pmb{\mu}^2(\pmb{\theta})
\end{align}
where the squaring operation in Equation (\ref{eq:cov}) is shorthand for the outer product.
In <a href="https://ui.adsabs.harvard.edu/abs/2019AJ....157...64L/abstract">Luger et al. (2019)</a> I showed that $\mathbf{f}$ may be computed from a
linear operation on the vector of spherical harmonic coefficients
describing the surface, $\mathbf{y}$:
\begin{align}
    \label{eq:fAy}
    \mathbf{f} = \mathbf{A} \, \mathbf{y}
    \quad,
\end{align}
where $\mathbf{A}$ is the <span style="font-style:italic">starry</span> design matrix, which is implicitly
a function of $\pmb{\theta}$ (as it depends on the stellar inclination
and rotation period, for example).
Given Equation (\ref{eq:fAy}),
we may write the mean and covariance of our flux GP as
\begin{align}
    \pmb{\mu}(\pmb{\theta})
     & = \mathbf{A}(\pmb{\theta}) \, \pmb{\mu}_{\mathbf{y}}(\pmb{\theta})
    \\
    \pmb{\Sigma}(\pmb{\theta})
     & = \mathbf{A}(\pmb{\theta}) \, \pmb{\Sigma}_{\mathbf{y}} \, \mathbf{A}^\top(\pmb{\theta})
    \quad,
\end{align}
where
\begin{align}
    \label{eq:mean_y}
    \pmb{\mu}_{\mathbf{y}}(\pmb{\theta})
     & = \mathrm{E} \Big[ \mathbf{y} \, \Big| \, \pmb{\theta} \Big]
    \\
    \label{eq:cov_y}
    \pmb{\Sigma}_{\mathbf{y}}(\pmb{\theta})
     & = \mathrm{E} \Big[ \mathbf{y} \, \mathbf{y}^\top \, \Big| \, \pmb{\theta} \Big] - \pmb{\mu}_{\mathbf{y}}^2(\pmb{\theta})
\end{align}
are the mean and covariance of the GP in the spherical harmonics basis.
It's a <span style="font-style:italic">lot</span> of math, but under certain
assumptions it's possible to
compute the expectations in the expressions above analytically. These
are given by the integrals
\begin{align}
    \label{eq:exp_y}
    \mathrm{E} \Big[ \mathbf{y} \, \Big| \, \pmb{\theta} \Big]
     & =
    \int \mathbf{y}(\mathbf{x} ) \, p(\mathbf{x} \, \big| \, \pmb{\theta})\mathrm{d}\mathbf{x}
    \\
    \label{eq:exp_yy}
    \mathrm{E} \Big[ \mathbf{y} \, \mathbf{y}^\top \, \Big| \, \pmb{\theta} \Big]
     & =
    \int \mathbf{y}(\mathbf{x} ) \mathbf{y}^\top(\mathbf{x} ) \, p(\mathbf{x} \, \big| \, \pmb{\theta})\mathrm{d}\mathbf{x}
    \quad,
\end{align}
where $\mathbf{x}$ is a random vector-valued variable corresponding to a particular
distribution of features on the surface  (i.e., the size and location of star spots)
and $p(\mathbf{x} \, \big| \, \pmb{\theta})$ is its probability density
function (PDF).
As we are specifically interested in modeling the effect of star spots
on stellar light curves, we let
\begin{align}
    \mathbf{x} = \left( \xi \,\, \lambda \,\, \phi \,\, \rho \right)^\top
\end{align}
and
\begin{align}
    \label{eq:RRs}
    \mathbf{y}(\mathbf{x}) =
    \xi
    \,
    \mathbf{R}_{\hat{\mathbf{y}}}(\lambda)
    \,
    \mathbf{R}_{\hat{\mathbf{x}}}(\phi)
    \,
    \mathbf{s}(\rho)
    \quad,
\end{align}
where $\xi$ is the contrast of a spot,
$\lambda$ is its longitude, $\phi$ is its latitude,
and $\rho$ is its radius.
The vector function $\mathbf{s}(\rho)$
returns the spherical harmonic expansion of a negative unit brightness
circular spot at $\lambda = \phi = 0$,
$\mathbf{R}_{\hat{\mathbf{x}}}(\phi)$ is the <a href="https://en.wikipedia.org/wiki/Wigner_D-matrix">Wigner matrix</a> that rotates the
expansion about $\hat{\mathbf{x}}$ such that the spot is centered at a
latitude $\phi$, and $\mathbf{R}_{\hat{\mathbf{y}}}(\lambda)$ is the Wigner
matrix that then rotates the
expansion about $\hat{\mathbf{y}}$ such that the spot is centered at a
longitude $\lambda$; these three functions are detailed in the sections below.
Equation (\ref{eq:RRs}) thus provides a way of converting a random variable
$\mathbf{x}$ describing the size, brightness, and position of a spot to the
corresponding representation in terms of spherical harmonics.
Note, importantly, that we are not interested in any specific value of
$\mathbf{y}$; rather, we would like to know its expectation value under
the probability distribution governing the different spot properties $\mathbf{x}$,
i.e., $p(\mathbf{x} \, \big| \, \pmb{\theta})$.
For simplicity, we assume that $p(\mathbf{x} \, \big| \, \pmb{\theta})$
is separable in each of the four spot properties:
\begin{align}
    p(\mathbf{x} \, \big| \, \pmb{\theta})
    =
    p(\xi \, \big| \, \pmb{\theta}_{\xi}) \,
    p(\lambda \, \big| \, \pmb{\theta}_{\lambda}) \,
    p(\phi \, \big| \, \pmb{\theta}_{\phi})\,
    p(\rho \, \big| \, \pmb{\theta}_{\rho})
    \quad,
\end{align}
where
\begin{align}
    \pmb{\theta} = \left(
    \pmb{\theta}_{\xi} \, \,
    \pmb{\theta}_{\lambda} \, \,
    \pmb{\theta}_{\phi} \, \,
    \pmb{\theta}_{\rho} \right)^\top
\end{align}
is the vector of hyperparameters of our GP.
This allows us to rewrite the expectation integrals (\ref{eq:exp_y})
and (\ref{eq:exp_yy}) as
\begin{align}
    \label{eq:exp_y_sep}
    \mathrm{E} \Big[ \mathbf{y} \, \Big| \, \pmb{\theta} \Big]
     & =
    \mathbf{e_4}(\pmb{\theta})
    \\[1em]
    \label{eq:exp_yy_sep}
    \mathrm{E} \Big[ \mathbf{y} \, \mathbf{y}^\top \, \Big| \, \pmb{\theta} \Big]
     & =
    \mathbf{E_4}(\pmb{\theta})
\end{align}
where we define the first moment integrals
\begin{align}
    \label{eq:e1}
    \mathbf{e_1}(\pmb{\theta}_\rho)
     & \equiv
    \int
    \mathbf{s}(\rho) \,
    p(\rho \, \big| \, \pmb{\theta}_{\rho}) \,
    \mathrm{d}\rho
    %
    \\[1em]
    %
    \label{eq:e2}
    \mathbf{e_2}(\pmb{\theta}_\phi, \mathbf{e_1})
     & \equiv
    \int
    \mathbf{R}_{\hat{\mathbf{x}}}(\phi) \,
    \mathbf{e_1} \,
    p(\phi \, \big| \, \pmb{\theta}_{\phi}) \,
    \mathrm{d}\phi
    %
    \\[1em]
    %
    \label{eq:e3}
    \mathbf{e_3}(\pmb{\theta}_\lambda, \mathbf{e_2})
     & \equiv
    \int
    \mathbf{R}_{\hat{\mathbf{y}}}(\lambda) \,
    \mathbf{e_2} \,
    p(\lambda \, \big| \, \pmb{\theta}_{\lambda}) \,
    \mathrm{d}\lambda
    \\[1em]
    \label{eq:e4}
    \mathbf{e_4}(\pmb{\theta}_\xi, \mathbf{e_3})
     & \equiv
    \int
    \xi \,
    \mathbf{e_3} \,
    p(\xi \, \big| \, \pmb{\theta}_{\xi}) \,
    \mathrm{d}\xi
    %
\end{align}
and the second moment integrals
\begin{align}
    \label{eq:E1}
    \mathbf{E_1}(\pmb{\theta}_\rho)
     & \equiv
    \int
    \mathbf{s}(\rho) \, \mathbf{s}^\top(\rho) \,
    p(\rho \, \big| \, \pmb{\theta}_{\rho}) \,
    \mathrm{d}\rho
    %
    \\[1em]
    %
    \label{eq:E2}
    \mathbf{E_2}(\pmb{\theta}_\phi, \mathbf{E_1})
     & \equiv
    \int
    \mathbf{R}_{\hat{\mathbf{x}}}(\phi) \,
    \mathbf{E_1} \,
    \mathbf{E_1}^\top \,
    \mathbf{R}_{\hat{\mathbf{x}}}^\top(\phi) \,
    p(\phi \, \big| \, \pmb{\theta}_{\phi})
    \mathrm{d}\phi
    %
    \\[1em]
    %
    \label{eq:E3}
    \mathbf{E_3}(\pmb{\theta}_\lambda, \mathbf{E_2})
     & \equiv
    \int
    \mathbf{R}_{\hat{\mathbf{y}}}(\lambda) \,
    \mathbf{E_2} \,
    \mathbf{E_2}^\top \,
    \mathbf{R}_{\hat{\mathbf{y}}}^\top(\lambda) \,
    p(\lambda \, \big| \, \pmb{\theta}_{\lambda})
    \mathrm{d}\phi
    \\[1em]
    %
    \label{eq:E4}
    \mathbf{E_4}(\pmb{\theta}_\xi, \mathbf{E_3})
     & \equiv
    \int
    \xi^2 \,
    \mathbf{E_3} \,
    \mathbf{E_3}^\top \,
    p(\xi \, \big| \, \pmb{\theta}_\xi)
    \mathrm{d}\xi
    %
    \quad.
\end{align}
The trick now is to find a functional form for the spherical harmonic expansion
of the spot, $\mathbf{s}(\rho)$, as well as for each of the probability
distributions
$p(\rho \, \big| \, \pmb{\theta}_{\rho})$,
$p(\phi \, \big| \, \pmb{\theta}_{\phi})$,
$p(\lambda \, \big| \, \pmb{\theta}_{\lambda})$,
and
$p(\xi \, \big| \, \pmb{\theta}_\xi)$,
such that the integrals above have closed-form expressions.
Note, importantly, that each integral is implicitly an integral of a vector
(in the case of the first moment integrals) or a matrix (in the case
of the second moment integrals), so we <span style="font-style=italic;">better</span>
find analytic expressions for them (otherwise we'll spend a very long time
integrating each of the indices of each vector and matrix numerically).

<br><br/>

I'm working on writing all this up into a paper, so I won't go any further into
the math in this blog post, other than to say that for a Lorentzian-like
star spot profile $\mathbf{s}(\rho)$, transformed Beta distributions for
$p(\rho \, \big| \, \pmb{\theta}_{\rho})$ and $p(\phi \, \big| \, \pmb{\theta}_{\phi})$,
and a log-Normal distribution for $p(\xi \, \big| \, \pmb{\theta}_\xi)$,
it is in fact possible to write down a closed-form solution to each of these
integrals. (Since we expect stars to be azimuthally symmetric on average,
$p(\lambda \, \big| \, \pmb{\theta}_{\lambda})$ is a uniform distribution over
$[0, 2\pi]$, so that's a relatively easy integral to take). If you're really
interested in the math behind all this, I'm working on the paper in
<a href="https://github.com/rodluger/starry_process">this github repository</a>,
with the always up-to-date manuscript draft
<a href="https://github.com/rodluger/starry_process/raw/master-pdf/tex/ms.pdf">here</a>.